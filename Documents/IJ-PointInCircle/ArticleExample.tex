\documentclass{InsightArticle}
\usepackage[dvips]{graphicx, mathtools}
\usepackage[dvips,
bookmarks,
bookmarksopen,
backref,
colorlinks,linkcolor={blue},citecolor={blue},urlcolor={blue},
]{hyperref}
\title{Exact Geometrical Predicate: Point in circle}

% 
% NOTE: This is the last number of the "handle" URL that 
% The Insight Journal assigns to your paper as part of the
% submission process. Please replace the number "1338" with
% the actual handle number that you get assigned.
%
\newcommand{\IJhandlerIDnumber}{1338}
\release{1.00}
\author{Bertrand Moreau, Alexandre Gouaillard}
\authoraddress{Singapore Immunology Network, Biopolis}
\begin{document}
\ifpdf
\else
   %
   % Commands for including Graphics when using latex
   % 
   \DeclareGraphicsExtensions{.eps,.jpg,.gif,.tiff,.bmp,.png}
   \DeclareGraphicsRule{.jpg}{eps}{.jpg.bb}{`convert #1 eps:-}
   \DeclareGraphicsRule{.gif}{eps}{.gif.bb}{`convert #1 eps:-}
   \DeclareGraphicsRule{.tiff}{eps}{.tiff.bb}{`convert #1 eps:-}
   \DeclareGraphicsRule{.bmp}{eps}{.bmp.bb}{`convert #1 eps:-}
   \DeclareGraphicsRule{.png}{eps}{.png.bb}{`convert #1 eps:-}
\fi
\maketitle
\ifhtml
\chapter*{Front Matter\label{front}}
\fi

\begin{abstract}
\noindent
This document describes the implementation in ITK of the "point in circle" geometrical predicate.
Based on Jonathan Shewchuk's work which implements an exact version of the predicate using standard
floating point types and arithmetic \cite{shewchuk97a}, the implementation consist of an ITK wrapper around the 
public domain C routines made available by the author of the precedent paper. 
\cite{shewchuk97aurl}
Wrapper using itk::PointSet, itk:CellInterface and itk:Mesh / itk:QuadEdgeMesh APIs are provided
along with corresponding examples which should provide enough details for users to directly
copy paste code in their application.

The application in mind for us is an exact and robust implementation of a delaunay triangulation /
voronoi tesselation in ITK, and will be presented in a separate paper.

\end{abstract}

\IJhandlenote{\IJhandlerIDnumber}
\tableofcontents
\pagebreak

\section{Description}
Considering a counterclockwise oriented circle represented by three points a,
b and c in a 2D space, a fourth point d lies inside the circle if and only if:

\begin{math}
det
\begin{bmatrix}
a_x & a_y & a_x^2+a_y^2 & 1 \\
b_x & b_y & b_x^2+b_y^2 & 1 \\
c_x & c_y & c_x^2+c_y^2 & 1 \\
d_x & d_y & d_x^2+d_y^2 & 1
\end{bmatrix}
>0
\end{math}

The orientation of the points a, b and c can be checked by considering the sign
of:

\begin{math}
det
\begin{bmatrix}
a_x & a_y & 1 \\
b_x & b_y & 1 \\
c_x & c_y & 1
\end{bmatrix}
\end{math}

If this determinant is positive, the points a, b and c are counterclockwise
oriented; if negative, they are clockwise oriented; and if equal to zero, they
are collinear.

Points a, b, c and d coordinates are exact precision numbers, as described in
\cite{shewchuk97a}. This test can be generalised to 3 or more dimensions.

For more details and illustrations, see here:
\url{http://www.cs.cmu.edu/%7Equake/robust.html}

\section{Implementation}

\section{Usage}

\bibliographystyle{plain}
\bibliography{InsightJournal}

\end{document}